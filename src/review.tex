\documentclass[a4paper,oneside,final,12pt,russian]{extarticle}
\usepackage[utf8]{inputenc}
\usepackage[T2A,T1]{fontenc}
\usepackage[russian]{babel}
\usepackage{vmargin}
\usepackage{indentfirst}
\setpapersize{A4}
\setmarginsrb{2cm}{1.5cm}{1cm}{1.5cm}{0pt}{0mm}{0pt}{13mm}

\usepackage{blindtext}
\usepackage{setspace}
\sloppy
\hyphenpenalty=5000

\begin{document}

\begin{center}
\textbf{ОТЗЫВ НАУЧНОГО РУКОВОДИТЕЛЯ}\\
\textbf{на выпускную квалификационную работу}\\
\textbf{студента К. Пучкова}\\
\textbf{<<Исследование сбора мусора в bitmap-индексах поисковых систем>>}\\
\end{center}

\onehalfspacing

\begin{flushleft}
Направление подготовки: 03.03.01 Прикладные математика и физика\\
Направленность (профиль) подготовки: Математическая физика,
компьютерные технологии и математическое моделирование в экономике\\
\end{flushleft}

Выпускная квалификационная работа Пучкова К. посвящена актуальной в условиях
быстрого роста количества хранимых и обрабатываемых данных проблеме очистки мусора в
битовых индексах, находящих сейчас самое широкое применение в поисковых движках и базах
данных. Основным результатом работы является описание нового алгоритма, позволяющего
ускорить поиск и уменьшить занимаемое индексом место за счёт удаления «повисших» ссылок:
теоретическое и эксперимента исследование его свойств. Постановка задачи работы
оправдана и актуальна, а ее результаты имеют практическую значимость для приложений,
осуществляющих поиск по данным.

К достоинствам работы относится теоретическая обоснованность полученных результатов
и тщательность экспериментального исследования. На основе проведенных вычислительных
экспериментов сделаны выводы о возможности использования алгоритмов в конкретных
прикладных задачах, выявлены их достоинства и недостатки. Студент продемонстрировал
способность разбираться в новых для себя темах, творчески мыслить и работать самостоятельно.

Выпускная квалификационная работа Пучкова К. соответствует требованиям, предъявляемым
к выпускным квалификационным работам бакалавра и может быть рекомендована к защите
с оценкой «отлично (9)», а её автор заслуживает рекомендации к поступлению в магистратуру МФТИ.

\singlespacing

\vspace{18mm}

\hspace{90mm}
\begin{minipage}{0.4\textwidth}
\begin{flushleft}
\textbf{Научный руководитель:}\\Неганов Алексей Михайлович\\
\vspace{4mm} \hrulefill \\
{\centering\scriptsize\textit{(подпись научного руководителя)}\\}
\end{flushleft}
\begin{flushright}
<<\rule{10mm}{0.4pt}>>\rule{30mm}{0.4pt} 2021 г.
\end{flushright}    
\end{minipage}

\thispagestyle{empty}
\end{document}
