\newpage
\section{Выводы}

Алгоритм сбора мусора оправдал теоретические ожидания эффективности времени
поиска и длительности исполнения по сравнению с алгоритмом <<мгновенного>>
удаления. 

Время поиска по всем вхождения ключа при фиксированном количестве
добавленных документов и размере блока дает выигрыш по сравнению
с поиском в индексе с мусором и алгоритмом <<мгновенного>> удаления при числе
добавленных документов $N \ge 10^3$. Данный результат получается для большинства
признаков с различной частотой встречаемости в документах.

Время поиска по первому вхождению ключа при фиксированном количестве
добавленных документов и размере блока не дает существенного выигрыша по
сравнению с поиском в индексе с мусором и алгоритмом <<мгновенного>> удаления
при числе добавленных документов $N \leq 10^5$. Данный результат логичен с точки
зрения устройства индекса.

Кроме того, наблюдается кратный выигрыш в скорости выполнения алгоритма сбора
мусора. Время работы выходит на линейный рост с малым наклоном при увеличении
числа документов $N$ в то время как алгоритм <<мгновенного>> удаления растет
экспоненциально.

В то же время предлагаемый метод не лишён важных недостатков. Хотя индекс,
построенный по этому принципу, и позволяет производить эффективный поиск по
всем вхождениям ключа, существенный разрыв с простым подходом проявляется лишь
при большом числе добавленных документов. Это соответствует реальным требованиями
к поисковым системам. Это также значит, что в случае, когда возможна реализация
случаев работы с малым числом данных, необходимо комбинировать два подхода для
различных целей.
