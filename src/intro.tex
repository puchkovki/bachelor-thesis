\newpage
\section{Введение}

\textbf{Актуальность темы исследования}

В 21-м веке важную роль в жизни человека играет интернет, а любое путешествие
по просторам всемирной паутины невозможно без специальных поисковых систем,
позволяющих получать запрошенную информацию. Первоочередной задачей
любой поисковой системы является быстрое предоставление пользователям
точной и уместной информации.

Современная эпоха характеризуется взрывным ростом числа хранимых и обрабатываемых
данных, что приводит к распространению новых парадигм в области хранения и
индексации данных. В силу активных разработок в области NoSQL-хранилищ множество
направлений исследования остаются незакрытыми \cite{No-SQL:IoT}.
В частности, остаётся малоизученной проблема сбора мусора в структурах данных
с требованием высокой скорости корректного чтения.

Дорогостоящая по времени и памяти операция удаления в поисковых структурах данных
приводит к тому, что немедленное удаление информации невыгодно поисковым системам
 — взамен документы часто становятся <<невидимыми>> для поисковых запросов
пользователя. Однако корректная политика хранения данных \cite{Data_Retention}
и продвигаемые повсеместно усиленные вариации <<права на забвение>> вынуждают
задуматься о качественном своевременном удалении требуемых данных.

В идеализированной системе поисковые индексы часто остаются неизменными. Однако
в реальности объекты часто становятся недоступными, что не представляет особой
угрозы, пока их мало. Большинство поисковых систем, даже такие повсеместно
используемые как Lucene \cite{Lucene:2008}, Elasticsearch\cite{Elasticsearch:2020}
и PostgreSQL\cite{GIN:2020}, обходятся этим случаем, лишь помечая удаленные
элементы отдельным битом. Фактическое удаление происходит лишь при слиянии
поискового дерева с SS-таблицами на диске.
Однако большое количество \textit{повисших ссылок} негативно влияет на
производительность системы. Данная проблема присуща поисковым
системам и до сих пор не была эффективно решена.

Кроме проблем с своевременным удалением также актуальна проблема безопасности.
Группа сотрудников Гарвардской школы права вместе с журналистами The Times определяли
уровень надёжности интернета как хранилища информации на примере ссылок в статьях
The New York Times \cite{NYT}.

В ходе исследования было рассмотрено более 553 тысяч статьей, которые содержали
внутри себя около 2,3 миллиона ссылок. Только за 3 последних года не менее 6\%
из них перестали работать, а если считать с 1998 года, то доля \textit{повисших}
ссылок в статьях превышает 72\%. Такие ссылки могут вести на сайт с
ошибкой \textbf{404 «Не найдено»} или перенаправлять на главную страницу целевого
сайта, но бывают варианты и похуже.

Вокруг \textit{повисших} ссылок на теневых ресурсах типа DarkNet выстроена целая
индустрия. Если ссылка ведёт на несуществующий документ, то его могут восстановить
с таким же доменным именем и адресацией до требуемой страницы. Таким образом
возрастает частота интернет мошенничества в сети \cite{Fraud}.

В данной работе рассматривается проблема эффективного удаления объектов из
NoSQL-хранилища с высокими требованиями к скорости чтения с предполагаемым
утверждением в релевантности полученных данных, что, таким образом, представляет
собой как открытую область теоретических исследований, так и практическую задачу
высокой значимости.

\textbf{Научная новизна}

Все основные результаты работы являются новыми. В работе предложен алгоритм
сбора мусора в поисковых системах, основанных на битовых индексах
поверх LSM-деревьев, использующих битовые инвертированные индексы для поиска и
хранения данных. Алгоритм сочетает в себе эффективность операций чтения с
зависящим от количества удалений записей временем выполнения запросов по
набору ключей при фиксированном количестве операций добавления и удаления
объектов.

\textbf{Теоретическая и практическая значимость}

Полученные в работе результаты имеют широкий спектр применений: полнотекстовый поиск,
обработка экспериментальных данных в физике высоких энергий, астрофизике, биологии,
моделировании климата и других естественных науках.

\textbf{Структура и объём научной работы}

Работа состоит из введения, постановки задачи, обзора литературы, 3
разделов, заключения и библиографии. Общий объем работы 48 страницы, из
них 28 страниц текста. Библиография включает 18 наименований на 2 страницах.
