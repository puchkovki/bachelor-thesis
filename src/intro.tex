\newpage
\section{Введение}

\textbf{Актуальность темы исследования.}

В 21-м веке огромную роль в жизни человека играет интернет, а любое путешествие
по просторам интернета невозможно без специальных поисковых систем, позволяющих
получать запрошенную и уместную информацию. Первоочередной задачей любой
поисковой системы является быстрое и корректное предоставление людям именно
той информации, которую они ищут.

Современная эпоха характеризуется взрывным ростом числа хранимых и обрабатываемых
данных, что приводит к распространению новых парадигм в области хранения и
индексации данных. В силу активных разработок в области NoSQL-хранилищ множество
направлений исследования остаются незакрытыми \cite{NeganovMastersthesis}.
В частности, остаётся слабоизученной проблема сбора мусора в структурах данных
с требованием высокой скорости корректного чтения.

Дорогостоящая операция удаления в поисковых структурах данных приводит к тому,
что удаление информации невыгодно поисковых системам — взамен они лишь делают ее
“невидимой” для поисковых запросов пользователя. Однако продвигаемые повсеместно
усиленные вариации “права на забвение” вскоре вынудят задуматься о качественном
своевременном удалении требуемых данных.

В идеализированной системе поисковые индексы остаются неизменными. Однако в
реальности объекты часто становятся недоступными, что не представляет особой
угрозы, пока их мало. Большинство поисковых систем, даже такие повсеместно
используемые, как Lucene \cite{Lucene:2008}, Elasticsearch\cite{Elasticsearch:2020}
и PostgreSQL\cite{GIN:2020}, обходятся этим случаем, лишь помечая элементы
удаленными отдельным битом. Фактическое удаление происходит лишь при слиянии
поискового дерева с SS-таблицами на диске.
Однако, большое количество \textit{“повисших” ссылок} негативно влияет на
производительность системы. Данная проблема присуща абсолютно всем поисковым
системам и до сих пор не была эффективно решена.

В данной работе рассматривается проблема эффективного удаления объектов из
NoSQL-хранилища с высокими требованиями к скорости чтения с предполагаемым
утверждением в релевантности полученных данных, что, таким образом, представляет
собой как открытую область теоретических исследований, так и практическую задачу
высокой значимости.

\textbf{Научная новизна.}

Все основные результаты работы являются новыми. В диссертации предложен новый
алгоритм сбора мусора в поисковых системах, основанных на  LSM с общими
деревьями, использующих битовые инвертированные индексы для поиска и хранения
данных. Алгоритм сочетает в себе эффективность операций чтения с зависящим от
количества удалений записей временем выполнения запросов по диапазону ключей
при фиксированном количестве операций добавления и удаления объектов.

\textbf{Теоретическая и практическая значимость.}

Полученные в диссертации результаты имеют широкий спектр применений.

\textbf{Структура и объём научной работы.}

Диссертация состоит из введения, постановки задачи, обзора литературы, Х
разделов, заключения и библиографии. Общий объем диссертации N страницы, из
них M страниц текста. Библиография включает A наименования на B страницах.