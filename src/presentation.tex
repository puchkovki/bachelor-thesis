\documentclass[aspectratio=169, pdf, 8pt, unicode]{beamer}
\usepackage[american,russian]{babel}
\usepackage[default]{sourcesanspro}
\usepackage{tikz}
\usepackage{tikzscale}
\usepackage{float}
\usepackage{graphicx}
\usepackage{pgfplotstable}
\usepackage{caption}
\usepackage{amsmath}
\usepackage{amssymb}
\usepackage{setspace}

\usetikzlibrary{chains,fit,shapes,arrows.meta}

\DeclareCaptionLabelFormat{gostfigure}{Рисунок #2}
\captionsetup[table]{labelsep=endash,justification=justified,singlelinecheck=false,font=normalsize,skip=0pt} 
\captionsetup[figure]{labelformat=gostfigure,labelsep=endash,justification=centering,singlelinecheck=false,font=normalsize} 
\pgfplotsset{compat=1.9}

\mode<presentation> {
\usetheme{Madrid}
}

\setbeamerfont{institute}{size=\normalsize}
\setbeamertemplate{itemize/enumerate body begin}{\large}
\setbeamertemplate{itemize/enumerate subbody begin}{\tiny}

\title[Бакалаврская работа]{Исследование сборки мусора в bitmap-индексах поисковых систем}

\author{Пучков Кирилл}

\institute[МФТИ]{
    Федеральное государственное автономное образовательное учреждение\\ 
    высшего образования\\
    <<Московский физико-технический институт (национальный исследовательский университет)>>\\
    Физтех-школа прикладной математики и информатики\\
    Кафедра теоретической и прикладной информатики\\
\vspace{0.5cm}
Научный руководитель --- А. М. Неганов
}

\date{Москва 2021}

\setbeamertemplate{caption}[numbered]

\begin{document}

\begin{frame}
\titlepage
\end{frame}

\begin{frame}
\frametitle{Содержание}
\tableofcontents
\end{frame}

\section{Постановка задачи}

\begin{frame}[fragile]
\frametitle{Постановка задачи}
\begin{figure}[H]
\centering
\caption{553 тысячи статей \textit{New York Times}}
\begin{table}[H]
\centering
\small
\singlespacing
\begin{tabular}{|c|c|}
    \hline
    Возраст ссылок      & Количество <<повисших>> ссылок, \%    \\ \hline
    Последние 3 года    & 6                                     \\ \hline
    С 1998 года         & $\geq$ 72                             \\ \hline
\end{tabular}
\end{table}
\end{figure}
\end{frame}

\begin{frame}[fragile]
\frametitle{Постановка задачи}
\begin{figure}[H]
\centering
\caption{Логическое представление индекса}
\begin{table}[H]
\centering
\small
\singlespacing
\begin{tabular}{|c|c|c|c|c|c|c|c|c|c|}
    \hline
                &\multicolumn{3}{c|}{block1}&\multicolumn{3}{c|}{block2}& \ldots \\ \hline
                & doc1  & doc2  & doc3      & doc4  & doc5      & doc6  & \ldots \\ \hline
    feature 1   & 0     & 0     & 0         & 1     & 1         & 1     & \ldots \\ \hline
    feature 2   & 1     & 0     & 1         & 0     & 1         & 1     & \ldots \\ \hline
    \vdots      & \vdots& \vdots& \vdots    & \vdots& \vdots    &\vdots & $\ddots$ \\ \hline
\end{tabular}
\label{index}
\end{table}
\end{figure}
\end{frame}

\begin{frame}[fragile]
\frametitle{Постановка задачи}
\begin{figure}[H]
\centering
\includegraphics[width=0.4\textwidth]{fig/parser.png}
\caption{Разбор поискового запроса}
\end{figure}
\end{frame}

\begin{frame}[fragile]
    \frametitle{Постановка задачи}
    {\large Задачи:}
    \vspace{5mm}
    \begin{enumerate}
    \item Исследовать существующие методы сбора мусора данных в поисковых системах и структурах данных.
    \vspace{5mm}
    \item Произвести анализ иных свойств метода, его преимуществ и недостатков.
    \vspace{5mm}
    \item Программно реализовать метод.
    \vspace{5mm}
    \item Сравнить скорость работы алгоритма с методом мгновенного удаления.
    \end{enumerate}
    \end{frame}

\section{Сбор мусора}

\begin{frame}[fragile]
\frametitle{Сбор мусора}

Карта \textit{danglingBmap} отображает блоки в индексе в битмапы,
содержащие биты повисших документов в блоке. В случае, если число повисших документов в блоке превышает пороговое — считаем блок <<мусорным>>.

\begin{block}{Алгоритм сбора мусора}
    \begin{enumerate}
        \item Для каждого блока в карте \textit{danglingBmap} проверить, превышает ли мощность битмапы допустимое пороговое значение для индекса.
        \item Для каждого такого блока выполнить операцию НЕ И с битмапой, полученной из первичного индекса. Полученная битмапа не содержит удаленных документов. Оставшиеся документы проверить во вторичном индексе на актуальность.
        \item Присвоить живым документам новый \textit{docId} и обновить на них ссылку во вторичном индексе.
        \item Продублировать значения битов в битмапах для новых \textit{docId} в первичном индексе.
        \item Удалить блок данных для старых \textit{docId}.
        \item В конце сбора мусора обнулить danglingBmap для <<мусорных>> блоков.
    \end{enumerate}
\end{block}
\end{frame}

\begin{frame}[fragile]
\frametitle{Сбор мусора}

Оценим теоретически возможное уменьшение числа блоков и, как следствие, ускорение поиска после сбора мусора.

\begin{block}{Теоретическое ускорение}
    Общий размер индекса
    \begin{equation}
        \left\lceil\frac{N}{\tau}\right\rceil \cdot F = \psi
    \end{equation}
    блоков, где $N$ — число добавленных документов, $\tau$ — размер блока, а
    $F$ — среднее число признаков на 1 документ.

    Таким образом, в случае плотных битмап, последовательного удаления и достаточного количества удаленных элементов $D \gg d$ ожидается, что в индексе образуется
    \begin{equation}
        \left\lceil\frac{D}{\tau}\right\rceil \cdot F = \zeta
    \end{equation}
    мусорных блоков, где $D$ — общее число удаленных документов, $d$ — пороговое <<мусорное>> значение для блока.

    Тогда после удаления <<мусорных>> блоков и завершения фоновых операций
    записи и слияния ожидается ускорение операции поиска признаков в 
    \begin{equation}
        \frac{\psi - \zeta}{\psi}
    \end{equation}
    раз.
\end{block}
\end{frame}

\begin{frame}[fragile]
\frametitle{<<Мгновенное>> удаление}

Создадим дополнительную структуру данных, карту \textit{doc2FieldFeature} , для отображения документов в набор их признаков и значений.

\begin{block}{Алгоритм слияния}
    \begin{enumerate}
        \item Результирующая битмапа удаленных документов == побитовое ИЛИ битмап удаленных документов для старого и нового элементов.
        \item Главная битмапа результирующего элемента == побитовое ИЛИ нового и старого элементов индекса. Далее к полученной битмапе
        применяется операция побитового НЕ И с полученной выше битмапой удаленных документов.
    \end{enumerate}
\end{block}
\end{frame}

\section{Экспериментальное исследование}

\begin{frame}[fragile]
\frametitle{Экспериментальное исследование}
\setstretch{1.5}

\begin{enumerate}
    \item Сравнить время поиска первого вхождения ключа до удаления,
    после сбора мусора и после <<мгновенного>> удаления
    \vspace{4mm}
    \item Сравнить время поиска всех вхождений ключа до удаления,
    после сбора мусора и после <<мгновенного>> удаления
    \vspace{4mm}
    \item Измерение времени работы алгоритма сбора мусора и алгоритма
    «мгновенного» удаления
    \vspace{4mm}
    \item Исследовать зависимость числа записей и слияний с внешней памятью
\end{enumerate}
\end{frame}

\begin{frame}[fragile]
    \frametitle{Экспериментальное исследование}
    \setstretch{1.5}
{\large Сценарии поиска:}
    \vspace{4mm}
\begin{enumerate}
    \item Запрос по всем вхождениям заданного ключа для фиксированного числа
    добавленных элементов $N_0$ и меняющегося размера блока битмапы $2^{x}$
    \vspace{4mm}
    \item Запрос по первому вхождению заданного ключа для фиксированного числа
    добавленных элементов $N_0$ и меняющегося размера блока битмапы $2^{x}$
    \vspace{4mm}
    \item Запрос по всем вхождениям заданного ключа для меняющегося числа добавленных
    элементов $N_0$
\end{enumerate}
\end{frame}

\begin{frame}[fragile]
\frametitle{Запрос по всем вхождениям ключа при фиксированном числе добавленных
документов и меняющемся размере блока}
\begin{figure}[H]
\centering
\includegraphics[width=0.5\textwidth]{fig/limit_1e6/1e5/to.png}
\caption{Зависимость времени поиска признака \textit{to} от размера блока для $10^5$ добавленных документов}
\end{figure}
\end{frame}

\begin{frame}[fragile]
\frametitle{Запрос по первому вхождению ключа при фиксированном числе
добавленных документов и меняющемся размере блока}
\begin{figure}[H]
\centering
\includegraphics[width=0.5\textwidth]{fig/limit_1/1e5/to.png}
\caption{Зависимость времени поиска признака \textit{to} от размера блока для $10^5$ добавленных документов}
\end{figure}
\end{frame}

\begin{frame}[fragile]
\frametitle{Запрос по всем вхождениям признака для меняющегося числа
добавленных элементов}
\begin{figure}[H]
\centering
\includegraphics[width=0.5\textwidth]{fig/to.png}
\caption{Зависимость времени поиска \textit{to} от количества добавленных документов}
\end{figure}
\end{frame}

\begin{frame}[fragile]
\frametitle{Запрос по всем вхождениям признака для меняющегося числа
добавленных элементов}
\begin{figure}[H]
\centering
\includegraphics[width=0.5\textwidth]{fig/from.png}
\caption{Зависимость времени поиска \textit{from} от количества добавленных документов}
\end{figure}
\end{frame}

\begin{frame}[fragile]
\frametitle{Сравнение времени работы алгоритма сбора мусора и алгоритма
«мгновенного» удаления}
\begin{figure}[H]
\centering
\begin{minipage}[h]{0.475\linewidth}
\includegraphics[width=1\textwidth]{fig/time_1e5.png}
\caption{Зависимость времени работы алгоритма от размера блока}
\end{minipage}
\hfil
\begin{minipage}[h]{0.35\linewidth}
\caption{Время работы алгоритмов для
    $10^5$ документов, мс}
\begin{table}[H]
      \centering
      \small
      \singlespacing
      \begin{tabular}{|p{1.5cm}|p{1.5cm}|p{2cm}|}
            \hline
            Размер блока & Сбор мусора                & <<Мгновенное>> удаление \\ \hline \hline
            10           & 2.54e-01                   & 2.26e+05              \\ \hline
            11           & 2.55e-01                   & 1.76e+05              \\ \hline
            12           & 2.54e-01                   & 3.14e+05              \\ \hline
            13           & 2.55e-01                   & 2.36e+05              \\ \hline
            14           & 2.55e-01                   & 2.33e+05              \\ \hline
            15           & 2.54e-01                   & 1.83e+05              \\ \hline
\end{tabular}
\end{table}
\end{minipage}
\end{figure}
\end{frame}

\begin{frame}[fragile]
\frametitle{Сравнение времени работы алгоритма сбора мусора и алгоритма
«мгновенного» удаления}
\begin{figure}[H]
\centering
\begin{minipage}[h]{0.475\linewidth}
\includegraphics[width=1\textwidth]{fig/time.png}
\caption{Зависимость времени работы алгоритма от количества добавленных документов}
\end{minipage}
\hfil
\begin{minipage}[h]{0.35\linewidth}
\caption{Время работы алгоритмов, с}
\begin{table}[H]
      \centering
      \small
      \singlespacing
      \begin{tabular}{|p{1.5cm}|p{1.5cm}|p{2cm}|}
        \hline
        Число документов & Сбор мусора                & <<Мгновенное>> удаление \\ \hline \hline
        $10^2$           & 3.60e-02                   & 1.71e-02              \\ \hline
        $10^3$           & 1.47e-01                   & 1.79e-01              \\ \hline
        $10^4$           & 2.54e-04                   & 5.11e+00              \\ \hline
        $10^5$           & 2.54e-04                   & 3.14e+02              \\ \hline
\end{tabular}
\end{table}
\end{minipage}
\end{figure}
\end{frame}

\begin{frame}[fragile]
\frametitle{Сравнение количества операций записи и слияния для различного числа
добавленных документов}
\begin{figure}[H]
\centering
\hfil
\includegraphics[width=0.6\textwidth]{fig/writecalls.png}
\caption{Зависимость количества операций записи на диск от количества добавленных документов}
\end{figure}
\end{frame}

\begin{frame}[fragile]
\frametitle{Сравнение количества операций записи и слияния для различного числа
добавленных документов}
\begin{figure}[H]
\centering
\hfil
\includegraphics[width=0.6\textwidth]{fig/merges.png}
\caption{Зависимость количества слияний с диском от количества добавленных документов}
\end{figure}
\end{frame}

\section{Выводы}

\begin{frame}[fragile]
\frametitle{Выводы}
\setstretch{2}
\begin{enumerate}
\item Время поиска по первому вхождению ключа уменьшается после сборки мусора
\item Время поиска по всем вхождениям ключа уменьшается после сборки мусора
\item Время поиска после сборки мусора меньше по сравнению с <<мгновенным>> удалением
\item Длительность сбора мусора в десятки тысяч раз меньше длительности <<мгновенного>> удаления
\item Алгоритм сбора мусора не менее эффективен в плане обращений к диску, чем алгоритм <<мгновенного>> удаления
\end{enumerate}
\end{frame}

\section{Направления дальнейшего исследования}

\begin{frame}[fragile]
\frametitle{Направления дальнейшего исследования}
\setstretch{2}
\begin{enumerate}
\item Оценка приоритета блоков при очистке для работы с флеш-памятью
\item Теоретическая оценка размера метаданных индекса при более общих
исходных предположениях
\item Детальное построение параллелизуемого алгоритма
\end{enumerate}
\end{frame}

\end{document}