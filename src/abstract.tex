\newpage
\section*{Аннотация}

\textbf{Цели и задачи работы}

Данная работа посвящена сбору мусора в поисковых системах, т.~е. эффективному и
своевременному удалению объектов, потерявших актуальность и/или указывающих на
несуществующие документы. \textit{Повисшей ссылкой} называется объект, который
более не доступен поисковой системе.

Целью данной работы является описание алгоритма сбора мусора в поисковой системе,
построенной на битовых инвертированных индексах. Каждому биту индекса
соответствует строка с индексируемым значением: его значение, равное 1, означает,
что запись, соответствующая позиции бита, содержит индексируемое значение для
данного столбца или свойства, 0 — что запись не содержит указанное значение.

Для определения эффективности предложенного алгоритма сравнивается скорость
поиска в поисковой системе и время работы самого алгоритма в сравнении с
поисковой системой, не накапливающей мусор. То есть удаляющей запрошенные
элементы немедленно после запроса.

\textbf{Полученные результаты}

В работе проанализированы существующие подходы к сбору мусора в поисковых
системах и структурах данных в контексте их применимости к решению поставленной
задачи.

Предложен алгоритм сбора мусора --- периодическая чистка индексов при условии
превышения количества \textit{повисших ссылок} в блоке индекса установленного
порогового значения.

Проведено теоретическое исследование свойств алгоритма, создана экспериментальная
реализация и исследовано ее поведение в сравнении с известными подходами для
некоторых сценариев использования.
