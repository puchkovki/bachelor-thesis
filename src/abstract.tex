\newpage
\section*{Аннотация}

\textbf{Цели и задачи работы.}

Данная работа посвящена сбору мусора в поисковых системах, т.~е. эффективному и своевременному
удалению объектов, потерявших актуальность и/или указывающих на несуществующие документы.
\textit{Повисшей ссылкой} называется объект, который более не доступен поисковой системе.

Целью данной работы является описание метода сбора мусора поисковой системы, построенных на
битовых инвертированных индексах, где каждому биту соответствует строка с индексируемым значением,
а его значение равное 1 означает, что запись, соответствующая позиции бита содержит индексируемое
значение для данного столбца или свойства.

Для определения качества предложенного алгоритма сравнивается скорость работы поисковой системы
по сравнению в поисковойо системой, использующей сбор мусора путем немедленного удаления
нерелевантных значений после аналогичного запроса.


\textbf{Полученные результаты.}

В работе проанализированы существующие подходы к сбору мусора в поисковых системах и структурах
данных в контексте их применимости к решению поставленной задачи.
Предложен новый метод сбора мусора --- периодическая чистка индексов при условии перехода общего
количества \textit{повисших ссылок} установленного порогового значения.
Проведено теоретическое исследование свойств алгоритма, создана экспериментальная реализация и
исследовано его поведение в сравнении с известными подходами для некоторых сценариев использования.