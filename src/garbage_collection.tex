\newpage
\section{Предложенное решение}

Bitmap-ы, хранящие в себе наличие либо отсутствие у документов определенных фич обычно сильно разрежены.
Поэтому они хранятся в сжатом виде, а для удобства использования поделены на блоки равного
размера, используя \textit{$roaring bitmaps$}. Сами данные в поисковом дереве расположены также по блокам,
которые однозначно определены отступом. В каждом блоке определим переменную \textit{dangling},
содержащую количество “повисших” документов в блоке. 

Процедура сбора мусора:
\begin{enumerate}
    \item Для каждого блока данных проверим счетчик
    \item При достижении порогового значения счетчика проверить остальные документы в блоке в \textit{LSM}
    индексе на валидность ключа
    \item Присвоить живущим документам новый \textit{DocId} и обновить на него ссылку в \textit{LSM}-индексе.
    \item Продублировать значения битов в bitmap-ах для нового \textit{DocId}.
    \item Обнулить блок данных для старого \textit{DocId}.
\end{enumerate}

Решение эффективно, но хотелось бы не просто выкидывать нулевые блоки из памяти, а по возможности переиспользовать их.
